
% \pagenumbering{arabic}
% \setcounter{page}{1}



% #TODO
% Jeremy's comments on the structure (TODO)
% It is pretty standard... Chapter 1 is an intro and then a chapter-by-chapter summary (basically a copy/paste of the abstract)
% 9:55
% chapter 2 is background info... individual chapters can have background that is "local" to that chapter, but for things that cut accross multiple chapters, place in Chapter 2
% 9:55
% then it is just a chapter for paper
% 9:55
% and then conclusions can be very sparse (e.g., two pages)
% 9:56
% basically what should the reader walk away with
% 9:56
% at the start of each chapter, state if it is published, who your co-authors are, and as needed, a sentence or two on what you specifically did as opposed to your co-authors
% 9:57
% in particular when a co-author is a student
% 9:57
% for me, you can say it was "supervised" by me



\chapter{Introduction} \label{sec:intro}

The impact of blockchain technology on the world is undeniable. It has been a decade since the first blockchain application, Bitcoin, was introduced to the world. Since then, the technology has been adopted by many industries and has been the subject of many research studies. The technology has been used in many applications, from digital cash (\eg Bitcoin~\cite{nakamoto2008bitcoin}), prediction markets~\cite{clark2014decentralizing}, and decentral governance~\cite{aragonwebsite}. Ethereum, one of the most popular blockchain platform that supports smart contracts, has been the most used platform for developing decentralized applications (DApps). Ethereum has been used in many applications, from decentralized exchanges (\eg Bancor~\cite{bancor}), crypto-collectibles (\eg CryptoKitties~\cite{cryptokitties}), gambling services (\eg Fomo3D~\cite{fomo3d}), and decentralized governance (\eg DAO).

Even though this technology is new and not fully matured, there are billions of dollars worth of assets stored and transacted on the blockchain, specifically Ethereum blockchain. A perception that crypocurrencies are criminal money has been formed in the public opinion. This perception is not without merit, as there are many cases of fraud and theft in the blockchain space. However, the blockchain technology is not inherently criminal. In fact, the technology has the potential to reduce the crime rate by providing a transparent and immutable ledger.

The main theme of this dissertation is to analyze a few different types of attacks on blockchain networks and applications, and provide solutions to mitigate some of the attacks. Many of these attacks we study are not new and have been studied in the traditional financial systems. However, the decentralized nature of the blockchain applications makes the attacks feasible by many actors and potentially more dangerous. 

%#TODO: more on the crime vs tech here


\section{Contributions}



\section{Outline}
The rest of the dissertation is organized as follows: In Chapter~\ref{sec:background}, we offer a concise introduction to the essential concepts necessary for comprehending this dissertation. In Chapter~\ref{sec:cryptojacking}, we study the cryptojacking concept, its fast paced growth, and the ethical questions surrounding this use of the blockchain technology. In Chapter~\ref{sec:frontrunning}, we study the frontrunning attacks on blockchain applications, provide a taxonomy of these attacks, and analyze the common mitigation methods. In Chapter~\ref{sec:oracles}, we study the oracle implementations on the blockchain, and the attacks on blockchain applications that use oracles. In Chapter~\ref{sec:auditing}, we study the auditing of blockchain-based assets and the systematic challenges a company might have to properly audit their crypto-assets. Lastly, in Chapter~\ref{sec:conclusion}, we provide some concluding remarks and future research prospects.