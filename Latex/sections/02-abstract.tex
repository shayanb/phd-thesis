% !TEX root = ../main.tex



\chapter*{\centering Abstract}

\begin{center}
    \Large
    {Uncovering Blockchain Challenges: \\ Technical Nuances and their Unforeseen Consequences}
        
    % \vspace{0.4cm}
    % \large
    % Thesis Subtitle
\end{center}

\noindent \textbf{Shayan Eskandari, Ph.D.} \\
\noindent \textbf{Concordia University, 2024} \\

\noindent In this dissertation, we explore the technical nuances of blockchain technology, its diverse applications, and the unforeseen consequences that have emerged. Cryptojacking, initially seen as a potential disruptor to the convoluted online advertising industry, ultimately succumbed to its own success due to regulatory gaps and technical intricacies. The inherent transparency and permissionless nature of blockchain allow every participant to potentially exploit privileged information --adversarial environment--, leading to front-running attacks and extracting value from users. Additionally, oracles, vital for providing real-world data to decentralized applications, might change the trust assumptions based on their implementations, which necessitate a deeper understanding of their operational mechanisms. With the rise of cryptoassets as a significant financial sector, auditors face challenges in accurately evaluating and reporting these assets due to their novelty. This dissertation aims to bridge the gap between the ambitious promises of blockchain technology and its real-world implications, highlighting the technical nuances that often lead to misunderstandings. We argue that by narrowing the divide between traditional regulatory view and blockchain's technological advancements, both auditors and companies working with cryptoassets stand to gain from the enhanced transparency and the potential for real-time information reporting (\eg financial statements) offered by this technology.









