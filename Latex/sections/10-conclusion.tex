% !TEX root = main.tex
\chapter{Concluding Remarks} \label{sec:conclusion}


The design of the blockchain technology and the applications of it can enable really novel approaches to remove trust in the intermediaries and significantly change the information flow in different businesses. However, this also brings forth some unforeseen consequences that were not possible before the existence of this technology.

In this thesis, we have explored some aspects of the blockchain technology and its applications, which as a consequence of the design and implementation of the technology, can lead to some unforeseen consequences. We have explored the following topics in this thesis:  


\begin{itemize}
    \item Chapter~\ref{sec:cryptojacking}: \textbf{Cryptojacking: from Replacing Ads to Invisible Abuse}: Browser-based mining promised to replace ads as a source of revenue for websites. However, it was abused by malicious actors and became a threat to the users. We have explored the different aspects of this threat in the wild and discussed possible solutions and ethical issues that might arise from the solutions.
    \item Chapter~\ref{sec:frontrunning}: \textbf{Blockchain Front-running: from Transparency to Extracting Value}: The blockchain technology promises to bring transparency to the financial markets. However, the transparency and permissionless aspect of the blockchains enables many new actors to observe the information flow and attempt to extract value from it. We showcase how some of these attacks happened in the past and discuss possible solutions to mitigate them. Furthermore, we discuss the ethical and legal issues that are left unanswered.
    \item Chapter~\ref{sec:oracles} \textbf{Oracles: from Ground Truth to Market Manipulation}: A blockchain, even though exists in an online decentralized network, it is still a closed system and cannot access the information from the outside world. Oracles are the entities that provide this information to the blockchain. However, the design of the oracles can lead to some unforeseen consequences and attacks that are foundational to the decentralized applications. We discuss the different types of oracles designs and implementations, and how they can be manipulated by malicious actors. 
    \item Chapter~\ref{sec:auditing} \textbf{Blockchain Audits: from Existence to Internal Controls}: Cryptoassets are assets that are mainly defined be some code that lives on the blockchain. The code is open source and anyone can read it. However, the code is not the only thing that matters in the financial statements of the company holding those cryptoassets. We discuss how auditors need to look at these assets, to be able to identify their existence and valuation, how the ownership can be proven and what it means to look at the internal control of the cryptoassets. Further we discuss some challenges and potential solutions to these problems. 
\end{itemize}


Each chapter in this thesis discusses their own future work and related \textit{follow-up work}. However, there are some overarching themes that can be explored further. %TODO: add more stuff here! 

\section{Future Work}
As we discussed in details in each chapter, there are many open questions and challenges that need to be addressed. Some of these challenges are technical in nature, some regarding user education on how to think about this new technology~\cite{gaggioli2019middleman}, while others are more related to the legal and ethical aspects of the blockchain technology integration with the current systems. In this section, we shortly discuss some of the future work.

Online advertisement is a multi-billion dollar industry. The current online advertisement ecosystem is broken, and it is not clear how to fix it. The current solutions are either not user-friendly or not privacy-preserving. Malvertising~\cite{li2012knowing} is a big threat to the users and given its complicated web of actors, it is almost impossible to change the current ecosystem. However, with blockchain technology we can change this ecosystem and bring transparency and fairness to the users. We have discussed some of the challenges and possible solutions in Chapter~\ref{sec:cryptojacking}. However, there are many open questions that need to be addressed. For example, how to design a system that is fair to the users and the publishers, while at the same time it is not vulnerable to the malicious actors? What would the user consent look like?, and many more.

As we further explained in Chapter~\ref{sec:frontrunning} and Chapter~\ref{sec:oracles}, the blockchain technology promises to bring transparency to the financial markets. However, the transparency and permissionless aspect of the blockchains enables many new actors to observe the information flow and attempt to extract value from it. Additionally, due to the close nature of blockchains, oracles are needed to bring the information from the outside world to the blockchain, and can be used to manipulate the market, either maliciously or unintentionally due to a technical bug. We discussed some potential solutions for front-running, which one of the most popular ones so far has been to embrace the front-running but ``democratize access'' to it by sharing the profit within the participating actors. This solution has gained a lot of traction in the blockchain community, however, it is not clear how to implement it in a way that is fair to all the participants. Furthermore, as we explored the early days of front-running in the traditional financial markets, we saw that the regulators were not able to keep up with the pace of the technology and the market. Similarly, It is not clear how the regulators can keep up with the pace of the MEV and the technical nuances of this new aspect of the blockchain technology.


Following the same pattern as the case studies in Chapter~\ref{sec:auditing}, there are many new possible approaches for financial statements, such as Real-time Financial Reporting, that were not possible to implement in a trust-minimized way before (See Section~\ref{sec:auditing:case-studies:valuation}). Although technically feasible, there are many nuances that changes the way auditors can trust and verify the information. Furthermore, the current accounting standards are not designed for this new technology, and it is not clear how to apply them to the blockchain technology, there are many examples in the chapter. Additionally, we still discussed a centralized company holding cryptoassets, however, many new entities in the ecosystem are DAOs (Decentralized Autonomous Organizations) that are not controlled by a centralized entity~\cite{tan2023open}. It is not clear how to audit these entities and what it means to audit these entities.

I hope this thesis can be a starting point for further research in this area. I believe that the blockchain technology can bring a lot of benefits to the society, however, it requires further research on ethical and fair design of the systems for all the participants that also does not lead to malicious unforeseen consequences.

% THE END


