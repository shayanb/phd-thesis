% !TEX root = main.tex

\chapter{Concluding Remarks} \label{sec:conclusion}


The design of the blockchain technology and the applications of it can enable really novel approaches to remove trust in the intermediaries and significantly change the information flow in different businesses. However, this also brings forth some unforeseen consequences that were not possible before the existence of this technology.

In this thesis, we have explored how the pseudonymity, transparency, and immutability of blockchain create both opportunities and challenges across different domains.  


\begin{itemize}
    \item Chapter~\ref{sec:cryptojacking}: \textbf{Cryptojacking: from Replacing Ads to Invisible Abuse}: In this chapter, we investigated the phenomenon of browser-based cryptojacking as an alternative to online advertisements. We detailed how this technology, initially proposed as a novel revenue generation method, was rapidly exploited by malicious actors. The chapter provided a comprehensive analysis of the ethical implications, the technical challenges, and potential mitigations for cryptojacking.


    \item Chapter~\ref{sec:frontrunning}: \textbf{Blockchain Front-running: from Transparency to Extracting Value}: In this chapter, we examined front-running attacks within blockchain ecosystems. We showcased how the transparency and decentralized nature of blockchains, intended to foster trust, can be manipulated for unfair advantages. Various strategies to mitigate these attacks were discussed, along with their ethical and legal implications.


    \item Chapter~\ref{sec:oracles} \textbf{Oracles: from Ground Truth to Market Manipulation}: In this chapter, we explored the critical role of oracles in connecting blockchains with external (real-world) data. We analyzed the vulnerabilities inherent in oracle designs and how these can be exploited for market manipulation and other malicious intents. Later we presented a modular framework for evaluating oracle design and reliability, and proposed mitigation strategies and their drawbacks.


    \item Chapter~\ref{sec:auditing} \textbf{Blockchain Audits: from Existence to Internal Controls}: In this chapter, we focused on the challenges of auditing blockchain-based assets. It addressed the complexities auditors face in verifying the existence, ownership, and valuation of cryptoassets. We go through four case studies and proposed solutions for improving audit processes and discussed the potential for developing specialized audit procedures and industry standards. 
\end{itemize}


Each chapter in this thesis discusses their own future work and related \textit{follow-up work}, as well as, address the research questions defined in~\ref{sec:research_questions}. 

\section{Revisiting the Research Questions}

All the research questions are addressed in their respective chapters. While some questions may not have definitive answers, they highlight important areas for future research. Below, we summarize the key points discussed for each research question:


\textbf{RQ1: How do the inherent characteristics of blockchain technology, such as transparency, pseudonymity, and immutability, create unique ethical challenges?}

The inherent characteristics of blockchain technology present distinct ethical challenges that are explored throughout this dissertation. Transparency, while enhancing trust and allowing for the verification of transactions, also poses significant privacy risks as transaction histories are publicly accessible, potentially exposing user activities and sensitive information. Additionally, the transparency in the network, creates front-running opportunities to extract value from unconfirmed transactions. Pseudonymity, which protects user identity, complicates accountability, making it difficult to trace illicit activities back to their perpetrators. Immutability, a cornerstone of blockchain's security, prevents the alteration or correction of transactions, even when they are fraudulent or flawed.

These aspects create a complex landscape for regulators and users, who must navigate the fine balance between leveraging blockchain's benefits and mitigating its risks. Regulators face the challenge of developing frameworks that protect user privacy without compromising the transparency that underpins blockchain's trust model. Users, on the other hand, need to understand how to engage with these technologies responsibly, ensuring that their actions do not inadvertently contribute to misuse or ethical breaches.

Throughout the thesis, these trade-offs and complexities are examined in various contexts, illustrating the multifaceted nature of blockchain ethics. This discussion provides a foundation for understanding how to approach blockchain technology in a safe and ethical manner, highlighting the need for ongoing dialogue and adaptive regulatory measures.


\textbf{RQ2: What are the ethical and practical implications of using browser-based cryptojacking as an alternative to traditional online advertisements?}

This question is comprehensively examined in Chapter~\ref{sec:cryptojacking}. Browser-based cryptojacking emerged as a novel method for generating revenue, potentially reducing privacy invasions associated with traditional online advertisements. However, it raises significant ethical concerns, primarily due to its tendency to operate without user consent, leading to the unauthorized exploitation of computational resources. The ethicality of cryptojacking is heavily dependent on transparency and obtaining explicit user consent, highlighting the urgent need for regulatory frameworks to address these issues.

Practically, cryptojacking has notable implications for user experience, device performance, and energy consumption. Unauthorized cryptojacking can degrade the performance of user devices, increase energy consumption, and potentially cause hardware damage. These practical concerns underscore the need for effective mitigation strategies. Existing solutions, such as browser extensions and ad-blocking software, offer some protection against cryptojacking but are not foolproof and do not address the root causes of the problem.

Future research in this area should prioritize the development of more robust mechanisms for detecting and preventing cryptojacking. This includes advancements in browser security features and more sophisticated detection algorithms. Additionally, exploring alternative revenue models that respect user privacy and autonomy is essential. Potential alternatives could involve more transparent and consensual forms of monetization, ensuring that users are fully aware and agreeable to the use of their computational resources. This comprehensive approach can help balance the benefits of innovative revenue models with the necessity of ethical practices and user protection.



\textbf{RQ3: How do front-running attacks exploit the decentralized nature of blockchain applications, and what strategies can mitigate these attacks?}

This question is thoroughly investigated in Chapter~\ref{sec:frontrunning}. Front-running attacks take advantage of the transparency and decentralized structure of blockchain applications. In decentralized blockchains, all transactions are visible to everyone before they are confirmed. This visibility allows (malicious) actors to observe pending transactions and insert their own transactions in a way that enables them to profit, often at the expense of others. 

The strategies to mitigate front-running attacks are diverse and multifaceted. One approach involves implementing privacy-enhancing technologies that obscure transaction details until they are confirmed, thus reducing the opportunity for front-runners to exploit transaction visibility. Another strategy is adjusting transaction ordering protocols to randomize or prioritize transactions in a manner that reduces the ability to predict and exploit the order of execution. Additionally, some propose democratizing the profit from front-running by redistributing gains among all participants, though this approach raises its own ethical and security concerns.

However, these mitigation strategies are not without challenges. Privacy-enhancing technologies can introduce new security vulnerabilities and might lead to increased complexity in the blockchain protocol. Adjusting transaction ordering protocols could potentially lead to censorship, where certain transactions are unfairly prioritized or delayed. The idea of democratizing profit redistribution can be contentious, as it might incentivize new forms of manipulation and raise questions about the fair distribution of rewards.

Future research should focus on developing and testing these mitigation strategies in real-world blockchain environments, ensuring they effectively reduce front-running without introducing new issues. Additionally, a thorough ethical analysis is required to balance the technical benefits of these strategies with their potential social and economic impacts. By addressing these concerns, the blockchain community can work towards creating a more secure and equitable ecosystem.


\textbf{RQ4: What role do oracles play in blockchain ecosystems, and how can their vulnerabilities be addressed to prevent manipulation?}

Chapter~\ref{sec:oracles} explores the crucial role of oracles in blockchain ecosystems and the significant vulnerabilities they might introduce. Oracles function as intermediaries that provide smart contracts with access to external data, allowing blockchain applications to interact with real-world events and information. This capability is crucial for a wide range of decentralized applications (DApps), from financial services to supply chain management, where real-time, accurate data is essential for decision-making and contract execution.

However, the reliance on oracles introduces a critical point of vulnerability. Oracles often depend on centralized data sources, which can be compromised, manipulated, or provide inaccurate information, posing substantial risks to the integrity of the blockchain applications. For instance, if an oracle is hacked or feeds false data, it can lead to incorrect contract execution, financial losses, and broader systemic risks within the blockchain network.

To address these vulnerabilities, several strategies are discussed in detail in Chapter~\ref{sec:oracles}. One approach is to decentralize data sources, thereby reducing the risk associated with any single point of failure. By aggregating data from multiple sources and employing consensus mechanisms among them, the accuracy and reliability of the data provided to the blockchain can be significantly enhanced. Another critical strategy involves implementing robust verification mechanisms. These can include cross-referencing data from different oracles, using trusted hardware to secure data feeds, and employing redundancy to ensure that data discrepancies can be detected and addressed promptly. Cryptographic proofs also play a vital role in ensuring the integrity and authenticity of data provided by oracles. Techniques such as zero-knowledge proofs can be used to verify the correctness of data without revealing the data itself, enhancing privacy and security. Additionally, secure multi-party computation (MPC) can enable multiple parties to jointly compute a function over their inputs while keeping those inputs private, further bolstering the trustworthiness of oracle-provided data.

Future research should focus on advancing these strategies and exploring new methodologies to enhance oracle security or remove the need of an oracle in the decentralized application. This includes developing standardized protocols for oracle implementation, enhancing the robustness of consensus algorithms used by decentralized oracles, and integrating advanced cryptographic techniques to safeguard data integrity. By addressing these challenges, the blockchain community can strengthen the reliability of oracles, thereby ensuring the secure and accurate execution of smart contracts and other blockchain applications.



\textbf{RQ5: What are the challenges faced by auditing firms in verifying cryptoassets, and how can these challenges be addressed to improve the audit process?}

Auditing cryptoassets presents significant challenges due to the need for technical expertise and the limitations of traditional auditing practices, which are not designed to accommodate the complexities of blockchain technology. Chapter~\ref{sec:auditing} explores these issues in depth, highlighting the specific difficulties auditing firms face when verifying the existence, ownership, and valuation of cryptoassets.

One major challenge is the lack of standardized accounting practices for cryptoassets. Current accounting standards were developed long before the advent of blockchain technology and thus are not equipped to handle its unique features. This gap creates uncertainty and inconsistency in how cryptoassets are recorded, valued, and reported. Additionally, the volatility and complexity of cryptoasset markets, aside from the underlying technology, further complicate the auditing process. Cryptoassets can experience rapid price fluctuations, which makes their valuation challenging and necessitates frequent updates to financial statements. Another significant challenge is the technical expertise required to understand blockchain technology and its implications for financial reporting. Auditors must be able to navigate the intricacies of blockchain transactions, including the verification of digital signatures, the assessment of smart contract functionalities, and in some cases understand the consensus mechanisms. This requires specialized training and a deep understanding of both blockchain technology and financial auditing principles.

To address these challenges, several solutions are proposed in Chapter~\ref{sec:auditing}. Developing specialized audit procedures tailored to the unique aspects of cryptoassets is essential. This includes creating new guidelines for the recognition, measurement, and disclosure of cryptoassets in financial statements. Enhancing technical capabilities within auditing firms is also crucial. This can be achieved through targeted training programs and the incorporation of blockchain analysis tools that can assist auditors in verifying transactions and assessing risks. Establishing industry standards for auditing cryptoassets is another critical step. These standards would provide a consistent framework for auditors to follow, reducing ambiguity and increasing the reliability of financial reports. Additionally, the adoption of modern and more reliable approaches to financial statements, such as Real-time Financial Reporting (RFR), can be enabled by blockchain technology. RFR allows for the continuous updating and verification of financial information, providing greater transparency and timeliness.

However, the current accounting standards must evolve to accommodate these new approaches. It remains unclear how to apply traditional accounting principles to blockchain technology, and there are numerous examples of these complexities discussed in the chapter. Furthermore, while the chapter focuses on centralized companies holding cryptoassets, the rise of Decentralized Autonomous Organizations (DAOs) presents new auditing challenges. DAOs operate without a central authority, making it difficult to determine how to audit these entities and what it means to conduct an audit in this context.

Future work should focus on developing robust frameworks for auditing DAOs and integrating blockchain technology into mainstream accounting practices. By addressing these challenges, the auditing profession can adapt to the evolving landscape of cryptoassets and ensure the integrity and accuracy of financial reporting in this dynamic sector.


\section{Discussion and Future Work}

As detailed in each chapter and through the research questions, numerous open questions and challenges remain. These challenges span technical, educational, psychological~\cite{gaggioli2019middleman}, legal, and ethical dimensions, necessitating a multifaceted approach to future research. In this section, we highlight additional areas for future work beyond those discussed in the previous sections.

The online advertisement industry, a multi-billion dollar sector, is currently plagued by inefficiencies, security and privacy concerns. The existing ecosystem is fragmented and often compromises user safety through practices like malvertising~\cite{li2012knowing}. Chapter~\ref{sec:cryptojacking} discussed some initial challenges and potential solutions of using blockchain technology to change this industry, but many questions remain unanswered. For example, how can we design a system that is equitable for both users and publishers while safeguarding against malicious actors? What mechanisms can ensure genuine user consent? Addressing these questions is crucial for developing a more user-friendly and privacy-preserving advertisement ecosystem.

Chapter~\ref{sec:frontrunning} and Chapter~\ref{sec:oracles} further explore how blockchain technology promises to enhance transparency in financial markets. However, this same transparency and the permissionless nature of blockchains also enable new forms of exploitation, such as front-running and market manipulation via oracles. While democratizing access to front-running profits has gained traction, its fair implementation remains a challenge. The rapid pace of technological advancement in blockchain often outstrips the ability of regulators to respond effectively, as seen historically with front-running in traditional financial markets. Future research should focus on developing fair and resilient systems to manage front-running and ensuring regulatory frameworks can adapt to the nuances of blockchain technology.

As examined in Chapter~\ref{sec:auditing}, innovative approaches to financial reporting, such as Real-time Financial Reporting (RFR), are now technically feasible thanks to the blockchain technology. These advancements could significantly enhance transparency and trust in financial statements. However, current accounting standards are not well-suited for blockchain's unique characteristics, posing challenges for auditors in verifying and trusting this information. Additionally, the rise of Decentralized Autonomous Organizations (DAOs) presents new auditing challenges, as these entities lack centralized control. Future work should aim to establish new auditing standards and methodologies tailored to the decentralized nature of DAOs and other blockchain-based entities~\cite{tan2023open}.

I hope this thesis serves as a starting point for further research in these areas. Blockchain technology holds immense potential to benefit society, but realizing these benefits requires continued exploration of ethical and fair system design. By addressing these challenges, we can mitigate malicious unforeseen consequences and harness blockchain's full potential.
