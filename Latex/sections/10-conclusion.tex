% !TEX root = main.tex
\chapter{Concluding Remarks} \label{sec:conclusion}


The design of the blockchain technology and the applications of it can enable really novel approaches to remove trust in the intermediaries and change the information flow in different businesses. However, this also brings forth some unforeseen consequences that were not possible before the existence of this technology.

In this thesis, we have explored some of aspects of the blockchain technology and its applications, which as a consequence of the design of the technology, can lead to some unforeseen consequences. We have explored the following topics in this thesis:  


\begin{itemize}
    \item Chapter~\ref{sec:cryptojacking}: \textbf{Cryptojacking: from Replacing Ads to Invisible Abuse}: Browser-based mining promised to replace ads as a source of revenue for websites. However, it was abused by malicious actors and became a threat to the users. We have explored the different aspects of this threat in the wild and discussed possible solutions and ethical issues that might arise from the solutions.
    \item Chapter~\ref{sec:frontrunning}: \textbf{Blockchain Front-running: from Transparency to Extracting Value}: The blockchain technology promises to bring transparency to the financial markets. However, the transparency and permissionless aspect of the blockchains enables many new actors to observe the information flow and attempt to extract value from it. We showcase how some of these attacks happened in the past and discuss possible solutions to mitigate them. Furthermore, we discuss the ethical and legal issues that are left unanswered.
    \item Chapter~\ref{sec:oracles} \textbf{Oracles: from Ground Truth to Market Manipulation}: A blockchain, even though exists in a online decentralized network, it is still a closed system and it cannot access the information from the outside world. Oracles are the entities that provide this information to the blockchain. However, the design of the oracles can lead to some unforeseen consequences and attacks that are foundational to the decentralized applications. We discuss the different types of oracles and how they can be manipulated. 
    \item Chapter~\ref{sec:auditing} \textbf{Blockchain Audits: from Existence to Internal Controls}: Cryptoassets are assets that are mainly defined be some code that lives on the blockchain. The code is open source and anyone can read it. However, the code is not the only thing that matters in a financial statements of the company holding those cryptoassets. We discuss how auditors need to look at these assets, to be able to identify their existence and valuation, how the ownership can be proven and what it means to look at the internal control of the cryptoassets. Further we discuss some of the challenges nad potential solutions to these problems. 
\end{itemize}


Each chapter in this thesis discusses their own future work and related \textit{follow-up work}. %However, there are some overarching themes that can be explored further. %TODO: add more stuff here! 


% “Research Agenda” subsection 
% —> circuling back to intro, talk about it at a greater detail, it’s ok to be repetative a bit to the intro parts, tie it back to the theme of the thesis. and add some future work (if it’s about technical nueances) 

% Conclusion chapter (~3 pages)
% - theme of the thesis
% - future work
% - conclusion on topics, like auditing stuff are cool, but how DAOs would look at that? 
% - Link back to papers and define what can be done for research
% - link back to intro and how the overarching theme is 


